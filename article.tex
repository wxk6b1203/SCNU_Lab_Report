% !TEX program = xelatex
\documentclass[10.5pt,a4paper,headings]{ctexart}
\usepackage{titling}
\usepackage{metalogo}

% \author{16-1-whoami?}
%
%Margin - 1 inch on all sides
%
\usepackage[letterpaper]{geometry}
\usepackage[OT2,T1]{fontenc}
\usepackage{inconsolata}
\usepackage{subfigure}
\usepackage{float}
\usepackage{rotating}
\usepackage{pdfpages}
\usepackage{csquotes}
\usepackage{svg}
\usepackage{minted}
\usepackage{bm}
\usepackage{xltxtra,xunicode}
\geometry{a4paper,top=1.0in, bottom=1.0in, left=1.0in, right=1.0in}


% array for tabular width
\usepackage{array}

% xeCJK package to resolve CJK font family by yourself.
% \usepackage{xeCJK}
    % \setCJKmainfont[BoldFont=SourceHanSerifSC-SemiBold, ItalicFont=STKaiti]{SourceHanSerifSC-Regular}
    % \setCJKsansfont[BoldFont=SourceHanSansSC-Bold]{SourceHanSansSC-Regular}
    % \setCJKmonofont{NotoSansMonoCJKsc-Regular}

% %

\ctexset{
    section/format=\bf\zihao{-4}\raggedright,
    subsection/format=\zihao{-4}\raggedright,
}

% times new roman for ascii
\usepackage{times}

%Doublespacing
\usepackage{setspace}
\doublespacing
%\onehalfspacing

 \usepackage{enumerate}

%Rotating tables (e.g. sideways when too long)
\usepackage{rotating}
\usepackage{fancyhdr}
\usepackage{indentfirst}
\usepackage{graphicx}
\usepackage{etoolbox}
\AtBeginEnvironment{minted}{\onehalfspacing
    \fontsize{10.5}{13}\selectfont}

    %
%Fancy-header package to modify header/page numbering (insert last name)
%
\pagestyle{fancy}
\lhead{}
\chead{}
\rhead{\XeLaTeX \ \ \thepage}
\lfoot{}
\cfoot{}
\rfoot{}
\renewcommand{\headrulewidth}{0pt}
\renewcommand{\footrulewidth}{0pt}

%To  sure we actually have header 0.5in away from top edge
%12pt is one-sixth of an inch. Subtract this from 0.5in to get headsep value
\setlength\headsep{0.333in}
\newcommand{\titleline}[1]{\hspace*{6em} #1 \\}
\newcommand*{\blankpage}{%
\vspace*{\fill}
{\centering\fontsize{32pt}{64pt} This page is intentionally left blank.\par}
\vspace{\fill}}
\makeatletter
\renewcommand*{\cleardoublepage}{\clearpage\if@twoside \ifodd\c@page\else
\blankpage
\thispagestyle{empty}
\newpage
\if@twocolumn\hbox{}\newpage\fi\fi\fi}

%
%Works cited environment
%(to start, use \begin{workscited...}, each entry preceded by \bibent)
% - from Ryan Alcock's MLA style file
%
\newcommand{\bibent}{\noindent \hangindent 40pt}
\newenvironment{workscited}{\newpage \begin{center} 参考文献 \end{center}}{\newpage }

% \newcommand{\fs}{\CJKfamily{fs}}

%
%Begin document
%
\begin{document}
\begin{titlepage}
\scshape


% Cover page
\begin{center}
\vspace*{0.6cm}
\includegraphics[width=0.65\textwidth]{./logo.jpg}

\doublespacing
\fontsize{18pt}{2em}\bf 
本科学生实验(实践)报告\\
\vspace{1em}
院系:计算机学院 
\hspace{2ex}
软件学院
\end{center}

\begin{flushleft}

\setlength{\parindent}{2em}
\bf
{\fontsize{14pt}{18pt}
% \doublespacing

\linespread{1.5}\selectfont
~\\
\titleline{实验课程:科技文献阅读与写作}

\titleline{实验项目:符合华南师大一般要求的实验报告封面}

\titleline{指导老师:PhD. Zhu}
~\\
\titleline{开课时间:2018 ~ 2019年度第 1 学期}

\titleline{专\hspace{2em}业:网络工程}

\titleline{班\hspace{2em}级:16级 1 班}

\titleline{姓\hspace{2em}名: 王 五}

\titleline{学\hspace{2em}号:20112101010}
}
\end{flushleft}

% Footer
\begin{center}
\vspace{3cm}
\fontsize{14pt}{21pt}
\bf 华南师范大学教务处
\end{center}

\end{titlepage}

\blankpage
\newpage
% table
\tableofcontents

\newpage

%%%%Title
\begin{center}
\zihao{3}
\tt 符合华南师大一般要求的实验报告封面
\end{center}
\vspace{6pt}
%%%%Changes paragraph indentation to 2em (2 CJK character)
\setlength{\parindent}{2em}

%%%%Begin body of paper here

%% abstract
\onehalfspacing
【摘要】

\doublespacing
\section{Section1}


\section{Section2}

\section{目录}

\section{代码}

\section{图片}

\section{表格}



\begin{workscited}

    \bibent
    参考文件引用了MLA style模板,如果不喜欢,可以注释掉并且使用\texttt{thebibliography}宏包。
    
    \bibent
    Baker, Gladys L., Wayne D. Rasmussen, Vivian Wiser, and Jane M. Porter. \textit{Century of Service: The First 100 Years of the United States Department of Agriculture.}[Federal Government], 1996. Print.
    
    \bibent
    Danhof, Clarence H. \textit{Change in Agriculture: The Northern United States, 1820-1870.} Cambridge: Harvard UP, 1969. Print.
    
    \bibent
    张三,李四:椰子的一百种吃法,《中国椰子协会》第一期02~12页
    
\end{workscited}

\end{document}
